% Options for packages loaded elsewhere
\PassOptionsToPackage{unicode}{hyperref}
\PassOptionsToPackage{hyphens}{url}
%
\documentclass[
]{article}
\usepackage{amsmath,amssymb}
\usepackage{iftex}
\ifPDFTeX
  \usepackage[T1]{fontenc}
  \usepackage[utf8]{inputenc}
  \usepackage{textcomp} % provide euro and other symbols
\else % if luatex or xetex
  \usepackage{unicode-math} % this also loads fontspec
  \defaultfontfeatures{Scale=MatchLowercase}
  \defaultfontfeatures[\rmfamily]{Ligatures=TeX,Scale=1}
\fi
\usepackage{lmodern}
\ifPDFTeX\else
  % xetex/luatex font selection
\fi
% Use upquote if available, for straight quotes in verbatim environments
\IfFileExists{upquote.sty}{\usepackage{upquote}}{}
\IfFileExists{microtype.sty}{% use microtype if available
  \usepackage[]{microtype}
  \UseMicrotypeSet[protrusion]{basicmath} % disable protrusion for tt fonts
}{}
\makeatletter
\@ifundefined{KOMAClassName}{% if non-KOMA class
  \IfFileExists{parskip.sty}{%
    \usepackage{parskip}
  }{% else
    \setlength{\parindent}{0pt}
    \setlength{\parskip}{6pt plus 2pt minus 1pt}}
}{% if KOMA class
  \KOMAoptions{parskip=half}}
\makeatother
\usepackage{xcolor}
\usepackage[margin=1in]{geometry}
\usepackage{color}
\usepackage{fancyvrb}
\newcommand{\VerbBar}{|}
\newcommand{\VERB}{\Verb[commandchars=\\\{\}]}
\DefineVerbatimEnvironment{Highlighting}{Verbatim}{commandchars=\\\{\}}
% Add ',fontsize=\small' for more characters per line
\usepackage{framed}
\definecolor{shadecolor}{RGB}{248,248,248}
\newenvironment{Shaded}{\begin{snugshade}}{\end{snugshade}}
\newcommand{\AlertTok}[1]{\textcolor[rgb]{0.94,0.16,0.16}{#1}}
\newcommand{\AnnotationTok}[1]{\textcolor[rgb]{0.56,0.35,0.01}{\textbf{\textit{#1}}}}
\newcommand{\AttributeTok}[1]{\textcolor[rgb]{0.13,0.29,0.53}{#1}}
\newcommand{\BaseNTok}[1]{\textcolor[rgb]{0.00,0.00,0.81}{#1}}
\newcommand{\BuiltInTok}[1]{#1}
\newcommand{\CharTok}[1]{\textcolor[rgb]{0.31,0.60,0.02}{#1}}
\newcommand{\CommentTok}[1]{\textcolor[rgb]{0.56,0.35,0.01}{\textit{#1}}}
\newcommand{\CommentVarTok}[1]{\textcolor[rgb]{0.56,0.35,0.01}{\textbf{\textit{#1}}}}
\newcommand{\ConstantTok}[1]{\textcolor[rgb]{0.56,0.35,0.01}{#1}}
\newcommand{\ControlFlowTok}[1]{\textcolor[rgb]{0.13,0.29,0.53}{\textbf{#1}}}
\newcommand{\DataTypeTok}[1]{\textcolor[rgb]{0.13,0.29,0.53}{#1}}
\newcommand{\DecValTok}[1]{\textcolor[rgb]{0.00,0.00,0.81}{#1}}
\newcommand{\DocumentationTok}[1]{\textcolor[rgb]{0.56,0.35,0.01}{\textbf{\textit{#1}}}}
\newcommand{\ErrorTok}[1]{\textcolor[rgb]{0.64,0.00,0.00}{\textbf{#1}}}
\newcommand{\ExtensionTok}[1]{#1}
\newcommand{\FloatTok}[1]{\textcolor[rgb]{0.00,0.00,0.81}{#1}}
\newcommand{\FunctionTok}[1]{\textcolor[rgb]{0.13,0.29,0.53}{\textbf{#1}}}
\newcommand{\ImportTok}[1]{#1}
\newcommand{\InformationTok}[1]{\textcolor[rgb]{0.56,0.35,0.01}{\textbf{\textit{#1}}}}
\newcommand{\KeywordTok}[1]{\textcolor[rgb]{0.13,0.29,0.53}{\textbf{#1}}}
\newcommand{\NormalTok}[1]{#1}
\newcommand{\OperatorTok}[1]{\textcolor[rgb]{0.81,0.36,0.00}{\textbf{#1}}}
\newcommand{\OtherTok}[1]{\textcolor[rgb]{0.56,0.35,0.01}{#1}}
\newcommand{\PreprocessorTok}[1]{\textcolor[rgb]{0.56,0.35,0.01}{\textit{#1}}}
\newcommand{\RegionMarkerTok}[1]{#1}
\newcommand{\SpecialCharTok}[1]{\textcolor[rgb]{0.81,0.36,0.00}{\textbf{#1}}}
\newcommand{\SpecialStringTok}[1]{\textcolor[rgb]{0.31,0.60,0.02}{#1}}
\newcommand{\StringTok}[1]{\textcolor[rgb]{0.31,0.60,0.02}{#1}}
\newcommand{\VariableTok}[1]{\textcolor[rgb]{0.00,0.00,0.00}{#1}}
\newcommand{\VerbatimStringTok}[1]{\textcolor[rgb]{0.31,0.60,0.02}{#1}}
\newcommand{\WarningTok}[1]{\textcolor[rgb]{0.56,0.35,0.01}{\textbf{\textit{#1}}}}
\usepackage{graphicx}
\makeatletter
\newsavebox\pandoc@box
\newcommand*\pandocbounded[1]{% scales image to fit in text height/width
  \sbox\pandoc@box{#1}%
  \Gscale@div\@tempa{\textheight}{\dimexpr\ht\pandoc@box+\dp\pandoc@box\relax}%
  \Gscale@div\@tempb{\linewidth}{\wd\pandoc@box}%
  \ifdim\@tempb\p@<\@tempa\p@\let\@tempa\@tempb\fi% select the smaller of both
  \ifdim\@tempa\p@<\p@\scalebox{\@tempa}{\usebox\pandoc@box}%
  \else\usebox{\pandoc@box}%
  \fi%
}
% Set default figure placement to htbp
\def\fps@figure{htbp}
\makeatother
\setlength{\emergencystretch}{3em} % prevent overfull lines
\providecommand{\tightlist}{%
  \setlength{\itemsep}{0pt}\setlength{\parskip}{0pt}}
\setcounter{secnumdepth}{-\maxdimen} % remove section numbering
\usepackage{bookmark}
\IfFileExists{xurl.sty}{\usepackage{xurl}}{} % add URL line breaks if available
\urlstyle{same}
\hypersetup{
  pdfauthor={Katie Willi},
  hidelinks,
  pdfcreator={LaTeX via pandoc}}

\author{Katie Willi}
\date{2025-06-03}

\begin{document}

\subsection{Review}\label{review}

In yesterday's lesson, we learned how to use several tidyverse functions
to describe our penguins data set. Let's review some of those functions
here.

First, we must load in the necessary packages for this lesson:

\begin{Shaded}
\begin{Highlighting}[]
\FunctionTok{library}\NormalTok{(tidyverse)}
\end{Highlighting}
\end{Shaded}

\begin{verbatim}
## Warning: package 'ggplot2' was built under R version 4.4.1
\end{verbatim}

\begin{verbatim}
## Warning: package 'purrr' was built under R version 4.4.1
\end{verbatim}

\begin{verbatim}
## Warning: package 'lubridate' was built under R version 4.4.1
\end{verbatim}

\begin{verbatim}
## -- Attaching core tidyverse packages ------------------------ tidyverse 2.0.0 --
## v dplyr     1.1.4     v readr     2.1.5
## v forcats   1.0.0     v stringr   1.5.1
## v ggplot2   3.5.2     v tibble    3.2.1
## v lubridate 1.9.4     v tidyr     1.3.1
## v purrr     1.0.4     
## -- Conflicts ------------------------------------------ tidyverse_conflicts() --
## x dplyr::filter() masks stats::filter()
## x dplyr::lag()    masks stats::lag()
## i Use the conflicted package (<http://conflicted.r-lib.org/>) to force all conflicts to become errors
\end{verbatim}

\begin{Shaded}
\begin{Highlighting}[]
\FunctionTok{library}\NormalTok{(palmerpenguins)}
\end{Highlighting}
\end{Shaded}

Next, we can load in the built-in penguins data set from
\{palmerpenguins\}:

\begin{Shaded}
\begin{Highlighting}[]
\FunctionTok{data}\NormalTok{(penguins)}
\end{Highlighting}
\end{Shaded}

We can use the \{tidyverse\} to answer questions like, ``Which penguins
species has the longest flipper length?''

\begin{Shaded}
\begin{Highlighting}[]
\NormalTok{longest\_flippers }\OtherTok{\textless{}{-}}\NormalTok{ penguins }\SpecialCharTok{\%\textgreater{}\%}
  \FunctionTok{group\_by}\NormalTok{(species) }\SpecialCharTok{\%\textgreater{}\%}
  \FunctionTok{summarize}\NormalTok{(}\AttributeTok{mean\_flipper\_length =} \FunctionTok{mean}\NormalTok{(flipper\_length\_mm, }\AttributeTok{na.rm =} \ConstantTok{TRUE}\NormalTok{)) }\SpecialCharTok{\%\textgreater{}\%}
  \FunctionTok{filter}\NormalTok{(mean\_flipper\_length }\SpecialCharTok{==} \FunctionTok{max}\NormalTok{(mean\_flipper\_length))}
\NormalTok{longest\_flippers}
\end{Highlighting}
\end{Shaded}

\begin{verbatim}
## # A tibble: 1 x 2
##   species mean_flipper_length
##   <fct>                 <dbl>
## 1 Gentoo                 217.
\end{verbatim}

We can make new columns with \texttt{mutate()}:

\begin{Shaded}
\begin{Highlighting}[]
\NormalTok{penguins }\OtherTok{\textless{}{-}}\NormalTok{ penguins }\SpecialCharTok{\%\textgreater{}\%}
  \FunctionTok{mutate}\NormalTok{(}\AttributeTok{flipper\_length\_cm =}\NormalTok{ flipper\_length\_mm}\SpecialCharTok{/}\DecValTok{10}\NormalTok{)}
\end{Highlighting}
\end{Shaded}

We can filter rows of the data frame. For example, filtering our
penguins dataset to only Adelie penguins on Torgersen Island:

\begin{Shaded}
\begin{Highlighting}[]
\NormalTok{adelie\_torgersen }\OtherTok{\textless{}{-}}\NormalTok{ penguins }\SpecialCharTok{\%\textgreater{}\%}
  \FunctionTok{filter}\NormalTok{(species }\SpecialCharTok{==} \StringTok{"Adelie"} \SpecialCharTok{\&}\NormalTok{ island }\SpecialCharTok{==} \StringTok{"Torgersen"}\NormalTok{)}
\end{Highlighting}
\end{Shaded}

We can select specific columns from the data frame. For example,
selecting the species and island columns from the penguins dataset:

\begin{Shaded}
\begin{Highlighting}[]
\NormalTok{penguins\_sub }\OtherTok{\textless{}{-}}\NormalTok{ penguins }\SpecialCharTok{\%\textgreater{}\%}
  \FunctionTok{select}\NormalTok{(island, species)}
\end{Highlighting}
\end{Shaded}

We can also use \texttt{mutate()} combined with \texttt{if\_else()} to
create a new variable based on conditions. For example, let's classify
penguins as having ``large'' or ``small'' body mass based on whether
they're above or below the median:

\begin{Shaded}
\begin{Highlighting}[]
\NormalTok{penguins }\OtherTok{\textless{}{-}}\NormalTok{ penguins }\SpecialCharTok{\%\textgreater{}\%}
  \FunctionTok{mutate}\NormalTok{(}\AttributeTok{body\_size =} \FunctionTok{if\_else}\NormalTok{(body\_mass\_g }\SpecialCharTok{\textgreater{}} \FunctionTok{median}\NormalTok{(body\_mass\_g, }\AttributeTok{na.rm =} \ConstantTok{TRUE}\NormalTok{), }
                            \StringTok{"large"}\NormalTok{, }
                            \StringTok{"small"}\NormalTok{))}
\end{Highlighting}
\end{Shaded}

\subsection{Visualization}\label{visualization}

An important part of data exploration includes visualizing the data to
reveal patterns you can't necessarily see from viewing a data frame of
numbers. Here we are going to walk through a very quick introduction to
\texttt{ggplot2}, using some code examples from the \{palmerpenguins\} R
package tutorial:
\url{https://allisonhorst.github.io/palmerpenguins/articles/intro.html}.

\{ggplot2\} is perhaps the most popular data visualization package in
the R language, and is also a part of the \{tidyverse\}. One big
difference about \texttt{ggplot} is that it \textbf{does not use the
pipe \texttt{\%\textgreater{}\%} operator} like we just learned, but
instead threads together arguments with \texttt{+} signs (but you can
pipe a data frame into the first \texttt{ggplot()} argument).

The general structure for ggplots follows the template below. Note that
you can also specify the \texttt{aes()} parameters within
\texttt{ggplot()} instead of your geom function, which you may see a lot
of people do. The mappings include arguments such as the x and y
variables from your data you want to use for the plot. The geom function
is the type of plot you want to make, such as \texttt{geom\_point()},
\texttt{geom\_bar()}, etc, there are a lot to choose from.

\begin{Shaded}
\begin{Highlighting}[]
\CommentTok{\# general structure of ggplot functions}
\FunctionTok{ggplot}\NormalTok{(}\AttributeTok{data =} \SpecialCharTok{\textless{}}\NormalTok{DATA}\SpecialCharTok{\textgreater{}}\NormalTok{) }\SpecialCharTok{+} 
  \ErrorTok{\textless{}}\NormalTok{GEOM\_FUNCTION}\SpecialCharTok{\textgreater{}}\NormalTok{(}\AttributeTok{mapping =} \FunctionTok{aes}\NormalTok{(}\SpecialCharTok{\textless{}}\NormalTok{MAPPINGS}\SpecialCharTok{\textgreater{}}\NormalTok{))}
\end{Highlighting}
\end{Shaded}

\textbf{Visualize variable distributions with
\texttt{geom\_historgram()}}

If you plan on doing any statistical analysis on your data , one of the
first things you are likely to do is explore the distribution of your
variables. You can plot histograms with \texttt{geom\_histogram()}

\begin{Shaded}
\begin{Highlighting}[]
\FunctionTok{ggplot}\NormalTok{(penguins) }\SpecialCharTok{+} 
  \FunctionTok{geom\_histogram}\NormalTok{(}\AttributeTok{mapping =} \FunctionTok{aes}\NormalTok{(}\AttributeTok{x =}\NormalTok{ flipper\_length\_mm))}
\end{Highlighting}
\end{Shaded}

\pandocbounded{\includegraphics[keepaspectratio]{03-data_exploration_lesson_files/figure-latex/unnamed-chunk-9-1.pdf}}

This tells us there may be a lot of variation in flipper size among
species. We can use the `fill =' argument to color the bars by species,
and \texttt{scale\_fill\_manual()} to specify the colors.

\begin{Shaded}
\begin{Highlighting}[]
\CommentTok{\# Histogram example: flipper length by species}
\FunctionTok{ggplot}\NormalTok{(penguins) }\SpecialCharTok{+}
  \FunctionTok{geom\_histogram}\NormalTok{(}\FunctionTok{aes}\NormalTok{(}\AttributeTok{x =}\NormalTok{ flipper\_length\_mm, }\AttributeTok{fill =}\NormalTok{ species), }\AttributeTok{alpha =} \FloatTok{0.5}\NormalTok{, }\AttributeTok{position =} \StringTok{"identity"}\NormalTok{) }\SpecialCharTok{+}
  \FunctionTok{scale\_fill\_manual}\NormalTok{(}\AttributeTok{values =} \FunctionTok{c}\NormalTok{(}\StringTok{"darkorange"}\NormalTok{,}\StringTok{"darkorchid"}\NormalTok{,}\StringTok{"cyan4"}\NormalTok{))}
\end{Highlighting}
\end{Shaded}

\pandocbounded{\includegraphics[keepaspectratio]{03-data_exploration_lesson_files/figure-latex/unnamed-chunk-10-1.pdf}}

Cool, now we can see there seems to be some pretty clear variation in
flipper size among species. Another way to visualize across groups is
with \texttt{facet\_wrap()}, which will create a separate plot for each
group, in this case species.

\begin{Shaded}
\begin{Highlighting}[]
\FunctionTok{ggplot}\NormalTok{(penguins) }\SpecialCharTok{+}
  \FunctionTok{geom\_histogram}\NormalTok{(}\FunctionTok{aes}\NormalTok{(}\AttributeTok{x =}\NormalTok{ flipper\_length\_mm, }\AttributeTok{fill =}\NormalTok{ species), }\AttributeTok{alpha =} \FloatTok{0.5}\NormalTok{, }\AttributeTok{position =} \StringTok{"identity"}\NormalTok{) }\SpecialCharTok{+}
  \FunctionTok{scale\_fill\_manual}\NormalTok{(}\AttributeTok{values =} \FunctionTok{c}\NormalTok{(}\StringTok{"darkorange"}\NormalTok{,}\StringTok{"darkorchid"}\NormalTok{,}\StringTok{"cyan4"}\NormalTok{)) }\SpecialCharTok{+}
  \FunctionTok{facet\_wrap}\NormalTok{(}\SpecialCharTok{\textasciitilde{}}\NormalTok{species)}
\end{Highlighting}
\end{Shaded}

\pandocbounded{\includegraphics[keepaspectratio]{03-data_exploration_lesson_files/figure-latex/unnamed-chunk-11-1.pdf}}

\textbf{Compare sample sizes with \texttt{geom\_bar()}}

Let's use ggplot to see sample size for each species on each island.

\begin{Shaded}
\begin{Highlighting}[]
\FunctionTok{ggplot}\NormalTok{(penguins) }\SpecialCharTok{+}
  \FunctionTok{geom\_bar}\NormalTok{(}\AttributeTok{mapping =} \FunctionTok{aes}\NormalTok{(}\AttributeTok{x =}\NormalTok{ island, }\AttributeTok{fill =}\NormalTok{ species))}
\end{Highlighting}
\end{Shaded}

\pandocbounded{\includegraphics[keepaspectratio]{03-data_exploration_lesson_files/figure-latex/unnamed-chunk-12-1.pdf}}

As you may have already noticed, the beauty about \texttt{ggplot2} is
there are a million ways you can customize your plots. This example
builds on our simple bar plot:

\begin{Shaded}
\begin{Highlighting}[]
\FunctionTok{ggplot}\NormalTok{(penguins, }\FunctionTok{aes}\NormalTok{(}\AttributeTok{x =}\NormalTok{ island, }\AttributeTok{fill =}\NormalTok{ species)) }\SpecialCharTok{+}
  \FunctionTok{geom\_bar}\NormalTok{(}\AttributeTok{alpha =} \FloatTok{0.8}\NormalTok{) }\SpecialCharTok{+}
  \FunctionTok{scale\_fill\_manual}\NormalTok{(}\AttributeTok{values =} \FunctionTok{c}\NormalTok{(}\StringTok{"darkorange"}\NormalTok{,}\StringTok{"purple"}\NormalTok{,}\StringTok{"cyan4"}\NormalTok{), }
                    \AttributeTok{guide =} \ConstantTok{FALSE}\NormalTok{) }\SpecialCharTok{+}
  \FunctionTok{theme\_minimal}\NormalTok{() }\SpecialCharTok{+}
  \FunctionTok{facet\_wrap}\NormalTok{(}\SpecialCharTok{\textasciitilde{}}\NormalTok{species, }\AttributeTok{ncol =} \DecValTok{1}\NormalTok{) }\SpecialCharTok{+}
  \FunctionTok{coord\_flip}\NormalTok{()}
\end{Highlighting}
\end{Shaded}

\pandocbounded{\includegraphics[keepaspectratio]{03-data_exploration_lesson_files/figure-latex/unnamed-chunk-13-1.pdf}}

This is important information, since we know now that not all species
were sampled on every island, which will have complications for any
comparisons we may want to make among islands.

\textbf{Visualize variable relationships with \texttt{geom\_point()}}

We can use \texttt{geom\_point()} to view the relationship between two
continuous variables by specifying the x and y axes. Say we want to
visualize the relationship between penguin body mass and flipper length
and color the points by species:

\begin{Shaded}
\begin{Highlighting}[]
\FunctionTok{ggplot}\NormalTok{(penguins) }\SpecialCharTok{+}
  \FunctionTok{geom\_point}\NormalTok{(}\AttributeTok{mapping =} \FunctionTok{aes}\NormalTok{(}\AttributeTok{x =}\NormalTok{ body\_mass\_g, }\AttributeTok{y =}\NormalTok{ flipper\_length\_mm, }\AttributeTok{color =}\NormalTok{ species))}
\end{Highlighting}
\end{Shaded}

\pandocbounded{\includegraphics[keepaspectratio]{03-data_exploration_lesson_files/figure-latex/unnamed-chunk-14-1.pdf}}

\end{document}
